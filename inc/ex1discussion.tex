\section{Discussion}

\todo{Look through experiment 1 notes in general for various insights and topics that can be discussed. Expect a good amount of discussion around the results and speculation on why they ended up as they did}


\subsection{Redirection Adaptability}
* Will probably be relevant to discuss
* Boelling et al. mentions there is some adaptivity effect at least for curvature gains. They would expect this to be true for other gains as well
* Do we see an adaptivity effect or not?
   * If no, it is likely that the alignment phases during AC2F limit adaptivity effects as gains are disabled at those times.
* The estimation method may play a role
* At the same time, once aligned AC2F disables gains completely, allowing the user to get used to normal head rotations again. This could counteract adaptivity effects

\subsection{Individual detection thresholds}
* Might be relevant to discuss
* Creating calibrating scenarios in software that allows users to estimate their own thresholds

\subsection{Large variability in detections}
* Can be many reasons for this
* One limitation is users looking left and right repeatedly makes them notice that they are being redirected
* S2C dampening helps with this after finishing a distractor, AC2F does not have an equivalent at the present as you need most of the redirection initially, then it is turned off once aligned


* There is definitely far less positive detections that negative ones, positive rot threshold is a tiny bit higher too (I think?)

\subsection{Participant Feedback on Pause - Turn - Centre}
* Generally positive
* It felt natural and easy to use
* Some mentioned that it took a little bit time to get used to though

\subsection{Comparing The Results With Fuglestad's Study}
\todo{Compare the difference in DT's between Bjørn's incremental and standard experiments. Argue that while the estimation method likely has an effect, so does the fact that users were distracted while performing the incremental experiment.}
