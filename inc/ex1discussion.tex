\section{Discussion}
This section focuses on discussing the presented results in Experiment 1.

\subsection{Rotation Gain Adaptation Effects}
It is interesting to see that there appears to be an adaptation effect for positive rotation gains, but not negative through the results. Research has already shown that adaptation effects can occur for curvature gains~\cite{bolling2019shrinking} and this has provided speculation as to whether this could be the case for other types of gains as well. The primary question lies in why the adaptation effect primarily happens with positive rotation gains and not negative. It may be that disabling redirection gains during alignment towards the centre has some effect on adaptation and as such, might somewhat obfuscate the full adaptation effect. In the case of this experiment, the estimation method might also have contributed to adaptation as it incrementally increases instead of randomly testing a range of gains. In order to more accurately test this adaptation effect, one approach could be to not disable gains during alignment. This approach would result in behaviour where gains dynamically change at a certain point from positive to negative and vice versa. The reasoning for this behaviour is that the algorithms will continuously overshoot and try to correct back towards alignment. As such, there may still be better approaches in terms of how to most accurately facilitate adaptation. 

It would be interesting to see future research on how this temporarily disabled gain affects adaptation effects. Another area that would be interesting to look deeper into is whether negative rotation gain adaptation is possible in specific scenarios. 

\subsection{Individuality of Detection Thresholds/Events}
When looking at the results of the experiment, one apparent detail that can be seen is how large the detection differences are on an individual level. This detail is in line with results from prior research~\cite{8446225, nguyen2018individual, schmitz2018you, fuglestad2018redirected} and should not come as particularly surprising. Despite this, the individuality of detection thresholds should be considered when developing redirected walking experiences. By creating short and effective detection threshold calibrations, it could be possible to provide an optimised experience for individuals. As such, it would be interesting to see further research into calibration approaches that are short enough for users to employ, while effective enough to estimate thresholds with strong accuracy. 

\subsection{Variability and Asymmetry in Detection Data}
In Figure~\ref{fig:rotationDetectionDataByParticipant}, we can see that there is a large amount of variability in detection events, even on an individual level. There can be many reasons for this, but there is one in particular that participants mentioned as making it very easy to notice. If participants keep looking left and right repeatedly, it becomes easy to notice that they are being redirected. The reasoning for this is that it becomes simpler to notice that the head does not return to the same orientation between the head turns. This issue is something that the S2C algorithm can mitigate through its dampening component whenever the participant is looking towards the centre of the physical space. AC2F as an algorithm does not necessarily have the same solution, but disabling redirection gains during alignment temporarily stops this scenario from being possible. In any case, this is not a particularly simple problem to solve, but it might be possible to attempt detecting when repeated left/right head turns are used and decrease or disable redirection. 

Other than variability, there is also an interesting asymmetry in terms of the number of detections that happen for positive and negative rotation gains. While it is not uncommon for positive rotation gain thresholds to be higher than negative ones in existing research, there is little to no information on the asymmetry of number of detections. This lack of information is primarily due to it not being possible to observe this effect when using the standard estimation method which the vast majority uses. It would be interesting to see further research into why the asymmetry is like this as positive rotation gains are harder to detect than negative ones in this case. This is some idle speculation, but it may be that the human brain is less susceptible to notice a redirection when they overshoot a target rotation, rather than undershooting it. Reaching the target rotation may be vital and as such, result in harder to notice redirection as the additional overshooting could find lower instinctive attention. In any case, further research into this topic would likely require more background within psychology and neuroscience to fully understand the asymmetric perception of gains. 

\subsection{Curvature Gain Detection Patterns}
Another interesting part of the results were the patterns of curvature detection events. When looking at Figure~\ref{fig:curvatureDetectionData}, we can see a very rigid ladder structure in terms of the detected curvature gains. Upon closer inspection, it is possible to see that this ladder effect comes from having a participant continuously press the detection button until they stop to notice the curvature gains. Each press will increase the curvature radius, and since all of these detections start at the minimum 2.5m radius, this ladder effect happens. The interesting thing is that the curvature gains are only detected when they are at the minimum radius while resulting in multiple detection events afterwards. It may be that simply noticing the curvature gains at one point increases the susceptibility to noticing it for a short time. This susceptibility increase would explain why participants need to go through multiple detection events until it becomes unnoticeable again. The fact that all of these detections start at a curvature radius of 2.5m before the resulting ladder effect is unfortunate as it means the cap for curvature gains was set too high. It might have been more ideal to put it to the same minimum radius as the redirected walking toolkit allows, which would be 1m. Doing so would be a lower risk in terms of potentially resulting in cybersickness than increasing caps for rotation gains. The reasoning for this is that the redirection effect is somewhat less immediate in strength due to slow walking speeds compared to head rotations, which may be rather fast. 

\subsection{Curvature Gain Adaptation}
Given that all curvature gain detections start at a curvature radius of 2.5m with some following detections, it is likely that adaptation plays some role in why the detections happen this late. The incremental gains style of estimation method means that curvature gains will gradually increase in strength. This incremental increase has been seen to cause adaptation effects in a study by Grechkin et al.~\cite{grechkin2016revisiting} and should be considered when looking at the estimated curvature gain threshold. From a practical point of view when working in redirected walking solutions, it could be useful to maximise the usage of this adaptation. Gradually increasing curvature gains until stopping at a strong, but set threshold could decrease the likelihood of detection. As a result, the effective redirection would be rather strong after some time has passed. The trade-off for this approach is of course that the effectiveness of curvature gains would be decreased until the target gain has been reached. Regardless, this could be a beneficial effect to consider.

\subsection{Comparing the Results With Fuglestad's Study}
Since Fuglestad performed a similar incrementing gains experiment in his research~\cite{fuglestad2018redirected}, it would be worthwhile to compare his results with the ones from Experiment 1. In his research, Fuglestad employed an abstract distractor while participants had to shoot a variety of moving targets with an ingame gun. Fuglestad's threshold results were provided with Steinicke et al.'s gain semantics~\cite{5072212}, so for the sake of comparison, they have been converted to the percentage-wise semantics of the redirected walking toolkit. Fuglestad observed a $0.835$ threshold for positive rotation gains and $-0.31$ threshold for negative rotation gains. In terms of negative rotation gain thresholds, the results are somewhat similar if we consider the $-0.3365$ threshold, which was seen during battles in Experiment 1. The reasoning for comparing with the threshold for battles is that this game state is the most similar to the abstract distractor that Fuglestad employed. The similarity in this case is that Fuglestad's incremental rotation gain experiment did not make use of any walking. 

The biggest difference is the positive rotation gain threshold as the one that was decided for use with Experiment 2 was a threshold of $0.6279$. This value reflects a considerable difference which warrants further discussion. The most apparent difference between Fuglestad's study and this experiment is that the redirection gain caps were set far higher. In Fuglestad's case, these were $4.0$ for positive rotation gains and $-0.9$ for negative. This is likely the largest contributor to this as Fuglestad's results showed many cases where positive rotation gain detections were in the higher end of the scale. This difference in caps would skew the data in a way that is impossible for Experiment 1 as a $1.0$ cap was used for positive rotation gains. As a side effect, the participants in Fuglestad's study experienced a fair amount of cybersickness. It should be noted that researching this was intentional though, as Fuglestad searched for differences between detection and discomfort thresholds.

Outside of the differences in redirection gain caps, there are also differences in terms of potential optical flow, tracking space size and potential distraction/engagement, which could have further effects. In any case, it is interesting to see that the results are at least similar in terms of negative rotation gains. Without being able to properly compare positive rotation gains though, there is limited ground to speculate about the cause of this effect. 

\subsection{Deciding on Which Estimated Thresholds to Use for Experiment 2}
Given that multiple thresholds were calculated, it is necessary to decide which of these will be used as the gain strengths in Experiment 2. For positive rotation gains, this is a rather simple choice as there was no significant difference between the walking and battle states. In this case, the mean threshold for all positive rotation gain detections is used ($0.6279$). For negative rotation gains though, there is a little bit more choice to consider.

While there is a significant difference between the walking and battle states, there is a substantial variability in the negative rotation gain detections. Given this large variability, it is rather hard to conclude how accurate the mean threshold would be. As such, the safest option in this case is to choose the lowest bounded threshold, which is the threshold for battles ($-0.3365$). The definition of ''safe'' in this case is to focus on minimising potential cybersickness symptoms of participants. 

Another option would be to dynamically change the strength of gains depending on which state of the game that participants are in. Given the results, this would only make sense for negative rotation gains, but it is something that could be considered. The current redirected walking solution in Ensemble Retriever does not have the functionality for switching between pairs of gains depending on the state of the game. Despite this, it should not be particularly challenging to implement as future work.

\subsection{Effect of AC2F Smoothing on Noticeability}
One area of the AC2F algorithm that definitely could see improvements is the smoothing component. Throughout Experiment 1, some participants mentioned that they noticed the slightly slippery smoothing that AC2F uses. The slippery effect in this case comes from AC2F smoothing from a rotation gain back towards normal head rotation whenever the user stops moving their head. AC2F's smoothing in this case works naturally when using one's body to rotate as the body elastically bobs slightly in the opposite direction after stopping. This elastic bob does not on the other hand occur as strongly if the user only rotates their head without rotating their body. This type of rotation creates a somewhat sliding smoothing effect as participants notice that they keep rotating slightly after they stop their head. 

Given that participants mentioned they noticed the smoothing effect at times, it might have had some effect on the estimated detection thresholds. Ideally, the smoothing algorithm should be able to smooth out changes between positive and negative rotation gains while still feeling natural to the user whenever they do smaller head rotations that do not move their body. One way to mitigate the slight slide in rotation that participants may notice when stopping their head rotation is to temporarily increase the speed of smoothing so it is not as noticeable. In the end though, there were not enough resources to implement this within the scope of this thesis. 

Given that smoothing components of redirection algorithms likely affect the noticeability of redirection, it presents some more opportunities for future research. It would be interesting to see some research on the noticeability effects of various smoothing approaches and what participants consider as most comfortable. Understanding what works best in terms of smoothing would be very useful for the sake of user comfort if redirected walking ever becomes more common for consumers. 

\subsection{Correlation Analysis Discussion}\label{sec:ex1demogDiscussion}
The primary insight that the correlation matrix points towards is there being some correlation between prior VR experience and the noticeability of redirection. It is hard to draw too much information out of the data, but it appears that the participants reporting the highest amount of VR experience also were among the most sensitive ones. Participants with no prior VR experience could be considered as among the most insensitive. This would be the case if we consider that several participants in this category did not detect any redirection at all. As for those in-between 0 and 6, there is a larger amount of variability in terms of sensitivity to redirection. While studies have shown that prior game experience has not significantly had any effect on noticeability~\cite{nguyen2018individual}, it would be interesting to see future research look deeper into how prior VR experience affects redirection sensitivity.
