\chapter{Method} \label{chap:generalmethods}
* This chapter consists of methods that are relevant for the thesis as a whole.
* Specific experimental methods can on the other hand be found respectively in Section X and Y 

\section{Literature Acquisition}
In order to find relevant literature for the literature analysis in Chapter~\ref{chap:relatedWork} a variety of search terms and databases were used. The list of keywords, keyword combinations and literature databases that were used can be found in Table~\ref{table:literaturekeywords}. 
Similar searches were also conducted with the ACM digital library database, but these searches did not provide any additional literature that had not already been found in other searches. As such, these queries are not included in the table. As a side note, these queries are a refreshed and combined version of previous literature searches in the IMT4205 - Research Project Planning and IMT4894 - Advanced Project Work courses. 

\begin{table}[tbh!]
\centering
\begin{tabularx}{\textwidth}{|m{2cm}|m{1.7cm}|m{2.7cm}|m{1.5cm}|m{1.20cm}|m{3.375cm}|} 
\hline
Keywords & Database & Combination & Filter & Results & Chosen\newline for Reading\\ 
\hline
"Redirected Walking"\newline
"Threshold"\newline
"Thresholds"\newline
"Comfort"& Google Scholar & ("Redirected Walking") AND ("Threshold" OR "Thresholds") AND "Comfort" & After 2014 & 69 & 7\\ 
\hline
"Redirected Walking"\newline
"Threshold"\newline
"Thresholds" & Google Scholar & ("Redirected walking") AND ("Threshold" OR "Thresholds") & After 2018 & 119 & 6\\
\hline
"Redirected Walking"\newline
"Threshold"\newline
"Thresholds" & IEEEXplore & "Redirected walking" AND ("Threshold" OR "Thresholds") & None & 13 & 5\\ 
\hline
"Redirected Walking"\newline
"Distractor"\newline
"Distractors" & Google Scholar & ("Redirected walking") AND ("Distractor" OR "Distractors") & After 2016 & 76 & 3\\
\hline
"Redirected Walking"\newline
"Distractor"\newline
"Distractors" & Google Scholar & ("Redirected walking") AND ("Distractor" OR "Distractors") & None & 172 & 3\\
\hline
"Virtual Reality"\newline
"Distractors"\newline
"Redirection" & Google Scholar & ("Virtual Reality" AND "Distractors" AND "Redirection") & None & 186 & 3\\
\hline
"Cinematic VR"\newline
"Attention" & Google Scholar & ("Cinematic VR" AND "Attention") & None & 123 & 2\\
\hline
\end{tabularx}
\caption{List over keywords and combinations that were used for the literature search}
\label{table:literaturekeywords}
\end{table}

\subsubsection{General Literature on Redirected Walking and Detection Thresholds}
For the first three queries in Table~\ref{table:literaturekeywords}, literature was picked for reading as long as the title or abstract focused on either detection thresholds or user comfort. The focus of the queries was to find recent development within the field of redirected walking and the estimation of detection thresholds. All result pages were scanned through in these three queries. 

\subsubsection{Distractors and Attention}
Table~\ref{table:literaturekeywords} also consists of three queries related to distractors. The first of these was conducted to look for state of the art applications of distractors while the second and third were used to acquire background literature on the topic. For the first search, all pages of query results were scanned through. For the second and third searches, the first ten pages of results were scanned through due to the higher sample. 

One final query related to cinematic VR and attention was added to look for related work in this field. The 10 first pages of results were scanned through for this query.
Literature was chosen for reading based on similar criteria as the queries in the previous section.

\subsubsection{Literature Acquired Through Citations}
As a secondary approach to acquiring literature, an additional \textbf{8} papers were found through citations in papers that were chosen in Table~\ref{table:literaturekeywords}. 

\subsubsection{Literature Acquired Through Discussions With Supervisors}
* Some additional papers were acquired through discussions with supervisors. 
* These are not directly related to the topic of distractors or redirected walking, but more so research papers providing an overview or details for various research methodologies. 
* Currently 1

This puts the total amount of literature that was acquired at \textbf{?} papers for the literature review in Chapter~\ref{chap:relatedWork}.

\section{GDPR, Data Collection}
* NSD response in appendix
* Particularly demographics questionnaire
* The software data collection can be mentioned as performance data
   * with some additional details of course
   
* Justification for each question and piece of data that is gathered

* Demographics questionnaire
   * Have you taken part in any prior experiments related to redirected walking?
      * To look for participants who might have training effects 
   * Gender
      * Might be worth asking
   * Age
      * In case age affects noticeability
   * Did you have to remove any optical corrections to use the HMD?
      * Could affect noticeability
   * How much previous experience do you have with virtual reality?
      * Could potentially have an effect on perception of virtual environments or cybersickness. Cybersickness tolerance in VR can be trained~\cite{hildebrandt2018get}
   * ID
      * Demographics papers stored in a locked box
   * As long as there are no ways to identify a person from here. 

* Send form to Simon after filling it in

* The papers with demographics answers is stored in a locked box 
* Experiment data can be anywhere
* Give the participants the ID in case they want to delete any data afterwards
   * GDRP compliance

* Did not ask for game experience as earlier research has suggested that this isn't relevant~\cite{nguyen2018individual}.

