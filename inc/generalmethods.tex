\chapter{General Methods} \label{chap:generalmethods}
This chapter consists of methods that are relevant for the thesis as a whole. Specific methods used for each of the two experiments can be found in Chapter~\ref{chap:ex1} and Chapter~\ref{chap:ex2}. 

\section{Literature Acquisition}
In order to find relevant literature for the literature analysis in Chapter~\ref{chap:relatedWork} a variety of search terms and databases were used. The list of keywords, keyword combinations and literature databases that were used can be found in Table~\ref{table:literaturekeywords}. 
Similar searches were also conducted with the ACM digital library database, but these searches did not provide any additional literature that had not already been found in other searches. As such, these queries are not included in the table. As a side note, these queries are a refreshed and combined version of previous literature searches in the IMT4205 - Research Project Planning and IMT4894 - Advanced Project Work courses. 

\begin{table}[tbh!]
\centering
\begin{tabularx}{\textwidth}{|m{2cm}|m{1.7cm}|m{2.7cm}|m{1.5cm}|m{1.20cm}|m{3.375cm}|} 
\hline
Keywords & Database & Combination & Filter & Results & Chosen\newline for Reading\\ 
\hline
"Redirected Walking"\newline
"Threshold"\newline
"Thresholds"\newline
"Comfort"& Google Scholar & ("Redirected Walking") AND ("Threshold" OR "Thresholds") AND "Comfort" & After 2014 & 69 & 7\\ 
\hline
"Redirected Walking"\newline
"Threshold"\newline
"Thresholds" & Google Scholar & ("Redirected walking") AND ("Threshold" OR "Thresholds") & After 2018 & 119 & 6\\
\hline
"Redirected Walking"\newline
"Threshold"\newline
"Thresholds" & IEEEXplore & "Redirected walking" AND ("Threshold" OR "Thresholds") & None & 13 & 5\\ 
\hline
"Redirected Walking"\newline
"Distractor"\newline
"Distractors" & Google Scholar & ("Redirected walking") AND ("Distractor" OR "Distractors") & After 2016 & 76 & 3\\
\hline
"Redirected Walking"\newline
"Distractor"\newline
"Distractors" & Google Scholar & ("Redirected walking") AND ("Distractor" OR "Distractors") & None & 172 & 3\\
\hline
"Virtual Reality"\newline
"Distractors"\newline
"Redirection" & Google Scholar & ("Virtual Reality" AND "Distractors" AND "Redirection") & None & 186 & 3\\
\hline
"Cinematic VR"\newline
"Attention" & Google Scholar & ("Cinematic VR" AND "Attention") & None & 123 & 2\\
\hline
\end{tabularx}
\caption{List over keywords and combinations that were used for the literature search}
\label{table:literaturekeywords}
\end{table}

\subsubsection{General Literature on Redirected Walking and Detection Thresholds}
For the first three queries in Table~\ref{table:literaturekeywords}, literature was picked for reading as long as the title or abstract focused on either detection thresholds or user comfort. The focus of the queries was to find recent development within the field of redirected walking and the estimation of detection thresholds. All result pages were scanned through in these three queries. 

\subsubsection{Distractors and Attention}
Table~\ref{table:literaturekeywords} also consists of three queries related to distractors. The first of these was conducted to look for state of the art applications of distractors while the second and third were used to acquire background literature on the topic. For the first search, all pages of query results were scanned through. For the second and third searches, the first ten pages of results were scanned through due to the higher sample. 

One final query related to cinematic VR and attention was added to look for related work in this field. The 10 first pages of results were scanned through for this query.
Literature was chosen for reading based on similar criteria as the queries in the previous section.

\subsubsection{Literature Acquired Through Citations}
As a secondary approach to acquiring literature, an additional \textbf{9} papers were found through citations in papers that were chosen in Table~\ref{table:literaturekeywords}. 

\subsubsection{Literature Acquired Through Discussions With Supervisors}
\textbf{3} additional research papers were acquired through discussions with the supervisors. These consisted of various research methodology papers as well as some pointers towards medical research that makes use of distractors in VR.

This puts the total amount of literature that was acquired at \textbf{41} papers for the literature review in Chapter~\ref{chap:relatedWork}.

\section{Personal Data Collection and GDPR Compliance}
Throughout the experiments, a demographics questionnaire was used to gather certain pieces of personal data. This section is dedicated to justify why each piece of personal data was collected and provide validation that the study itself is GDRP compliant. 

\subsection{Demographics Questionnaire}
The demographics questionnaire that was employed in both experiments can be found in Appendix~\ref{app:demographicsQuestionnaire}. It consisted of various demographical questions as well as optional qualitative feedback questions. The reasoning for using the following demographical questions is as follows:

\begin{description}
   \item[Gender:] Given that there is prior research on the potential effects of gender on sensitivity to redirection~\cite{nguyen2018individual}, this is a relevant piece of data to collect. 
   \item[Age Range:] There may be cognitive differences between younger and older participants that can affect the result. Age ranges are used instead of direct age so that recognising individuals is more difficult. 
   \item[Whether the participant needed to remove any optical corrections while in VR:] If a participant needs to remove any optical corrections to make use of a VR HMD, the sharpness of their vision may or may not be compromised. This could in turn have some effects on their sensitivity to redirection as it for example could affect the perceived optical flow. 
   \item[Whether the participant has taken part in prior redirected walking experiments:] This question was asked in case there are any trainable effects towards redirection. This could be the case as there may be similar effects to the possibility of mitigating cybersickness through training~\cite{hildebrandt2018get}. Furthermore, research suggests that adaptation effects exist for curvature gains~\cite{nguyen2018individual}.
   \item[How much prior VR experience the participants considers they have:] In a similar vein to the previous question, prior experience with VR could have some effect on how comfortable participants are in VR. This could in turn have some effects on how comfortable they are with using redirected walking.
\end{description}

One question that might have been relevant would be how much prior game experience participants have had. This was included in the questionnaire as research has suggested that this does not factor into the noticeability of redirection~\cite{nguyen2018individual}.

\subsection{GDPR Compliance and Data Anonymity}
In order to validate that this study was GDPR compliant and following other relevant regulations, a application form was sent to the Norwegian Centre for Research Data\footnote{\url{https://nsd.no/nsd/english/index.html}} (NSD). Their response, showing that the study is compliant with GDPR as well as their own terms can be seen in Appendix~\ref{app:nsdApproval}. The following two paragraphs consist of a summary of the information that was provided in the application to NSD.

The demographics data that was gathered made use of a paper questionnaire and required written consent. These two were tied together with a numerical ID which could be used if necessary to remove any personal data if requested. Participants were asked at the end of each experiment if they wished to be given their ID so it could be used for GDPR related requests. The demographics data and written consent were stored in a secure locked box at campus until data processing was necessary. Once processing was necessary, the paper data was transcribed and stored on the author's private home directory at NTNU's servers\footnote{\url{https://innsida.ntnu.no/web/guest/wiki/-/wiki/English/Your+files+on+the+NTNU+server}}. This data was then processed using NTNU's software farm\footnote{\url{https://innsida.ntnu.no/wiki/-/wiki/English/software+farm}}. The demographical data on the private home directory as well as the paper information are planned to be destroyed on the thesis hand-in date.

Outside of the demographics information that was recorded, each of the two experiments recorded a variety of software side performance data. This data was also tied to the demographics information with a numerical ID, but stored in the project repository on GitHub~\cite{projectRepository}. This data is openly available as there is no way to tie it to any individuals without access to the demographics information.

\section{Development Environment}
The development environment that was used to develop the Ensemble Retriever VR game consists of: 

\begin{itemize}
    \item Unity Engine, version 2018.3.5f1.
    \item Microsoft Visual Studio 2017 Integrated Development Environment.
    \item SteamVR Unity plugin for virtual reality development.
    \item A Mersenne Twister library~\cite{MersenneTwisterLibraryLink} for cases where high quality randomness is necessary.
    \item Azmandian et al.'s Redirected Walking Toolkit~\cite{azmandian2016redirected} for providing base redirected walking functionality.
\end{itemize}

Further details on Ensemble Retriever and additions to the redirected walking solution can be seen in Chapter~\ref{chap:implementation}.

\section{Hardware Environment for Experiments}
Throughout the two experiments, a desktop computer with the following specifications was used:
\begin{description}
   \item[CPU:] Intel i7-6700k
   \item[GPU:] Nvidia Geforce GTX 1080
   \item[RAM:] 16 GB
   \item[Operating System:] Windows 10 Pro
\end{description}

Together with this desktop computer, a HTC Vive HMD + Vive Controllers was employed together with 5m cable extensions to allow for full traversal of the physical tracking space without any major tethering issues. The physical space that was used can be seen in Section~\ref{sec:ex1physicalRoom}. Finally, a pair of Audio Technica ATH-MSR7BK headphones were used to provide sound for participants. 

\section{Software Environment for Experiments}
On the software side, the developed ''Ensemble Retriever'' game was used while running in the Unity Editor. The reasoning for using the software in the editor rather than as a built version was that the employed version of Unity had build times that took several hours. This would be rather inflexible in case any apparent issues appear during experiments that require rapid fixing. Instead, a more flexible option was used by running the game in the Unity Editor, allowing for small changes or fixes to be quickly implemented if needed. This does come with a small trade-off in terms of ingame performance as the editor results in higher overhead, but this was considered as an acceptable trade. 