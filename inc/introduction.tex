\chapter{Introduction}
\label{chap:introduction}

\section{Topic Covered by The Thesis}
This thesis covers the topic of redirected walking and a particular subarea in this field known as virtual distractors (or simply distractors). A problem with current day room scale solutions in virtual reality is that we often do not have a large amount of space to move around in for these types of experiences. Redirected walking aims to mitigate this problem by unnoticeably redirecting the user while they walk around to create the illusion of a fully explorable virtual world~\cite{razzaque2001redirected}. By doing so, it is possible to make better use of the available physical space while still creating an immersive experience. Despite this, redirected walking by itself is not sufficient enough to properly redirect the user in smaller physical spaces~\cite{5072212,azmandian2015physical}. By engaging the user with distractors on the other hand, it is possible to increase the degree of unnoticeable redirection while still keeping a high subjective sense of presence~\cite{peck2009evaluation}. These could be anything from activities in the virtual world to objects that can keep the user's attention.  


\section{Keywords}
Virtual Reality, Redirected Walking, Distractor, Distractors, Immersion, Subjective Sense of Presence, Computer Games, Games, Noticeability, Effectiveness, Detection, Detection Threshold, Detection Thresholds

\section{Problem Description}
Redirected walking by itself achieves full effectiveness in very large rooms which are unfeasible for most users. As an example: it is necessary to have a room that can fit a circle with a radius of 22 meters to entirely redirect the user in an unnoticeable manner when using one type of redirection technique~\cite{5072212, azmandian2015physical}. This is not only unfeasible for the average end-user to have, but also unfeasible for many modern head-mounted displays as it is challenging to track areas of this size. In smaller physical spaces, the user is expected to be told by the software to reorient themselves a fair amount whenever they are close to the physical walls. These reorientation events can break the user's subjective sense of presence and does not necessarily contribute to a good virtual experience. 

\section{Justification, Motivation and Benefits}
The limitations of physical space mean that it is all the more important to make sure that reorientation events are as effective and unintrusive to the user as possible~\cite{azmandian2015physical}. The primary benefits of a good redirected walking solution lie with the end-user as it allows them to experience virtual reality in a more immersive manner as well as providing lower amounts of cybersickness compared to other forms of locomotion~\cite{razzaque2001redirected, peck2011evaluation}. It allows the user to walk around in a virtual world that is larger than their available physical space, which could be seen as a crucial part of the experience. It also provides some benefits to the developers of VR software as they do not have to rely on the limits of physical space to the same degree as they currently do. While there has been some research on the topic of distractors over the years, it is not a large field of research. As such, it would be beneficial to provide some additional research to this field as the problem of limited physical space will not necessarily disappear in the near future. The usage of distractors for the sake of reorientation is particularly useful as they focus on reorienting in a manner that aims to provide lower amounts of intrusion into the experience compared to other approaches.

\section{Research Questions}\label{research:questions}
One area in particular that research on distractors in redirected walking does not seem to have delved much into is measuring how they affect the noticeability of redirection. Furthermore, if the highest unnoticeable redirection gains with distractors were estimated, it would also be interesting to see how effective the redirection could be in the context of a virtual environment like a game. The reason for using games as a context is that their interactive nature allows to broaden the design space of distractors in a manner that could be more engaging for the user. As such, the following research questions have been established:
\begin{description}
\item[$R_1$: ] How noticeable is redirected walking with distractors in a playful virtual environment?
\item[$R_2$: ] Given the highest unnoticeable gains, how effective is redirected walking with distractors in a playful virtual environment?
\end{description}

\section{Contributions}
The master thesis has a variety of contributions to redirected walking research. The following paragraphs provide a summary of what the thesis contributes.

\subsection{Providing an Example of State of the Art Distractor Usage}
First of all, the thesis aims to provide an example of how noticeable redirected walking is when using state of the art distractors. The reason for mentioning it will provide an example is that the design space for distractors is vast and as such, some implementations might be more effective than others. 

\subsection{Contributing to a Young Field of Research}
Furthermore, the thesis aims to contribute to a field of research that currently is reasonably small. Due to the size of this field, it would be beneficial to provide additional results that can point towards the effectiveness and noticeability of redirected walking with distractors. By doing so, it is possible to validate existing research as well as providing data that can be used for consideration by other researchers with interest in the topic. 

\subsection{Providing New Insights With Exploratory Methods}
Given that the research field on distractors is still relatively small, this study makes use of rather exploratory approaches and methods. This contributes a large amount of potential future work for researchers as well as insights which might not be possible to find with pre-established methods. As such, there is a significant focus on reproducibility to ensure that the exploratory methods that have been applied can be reproduced and reused by future research. 

\subsection{An Openly Available VR Game: ''Ensemble Retriever''}
This thesis also contributes an openly available VR game, titled ''Ensemble Retriever'' which makes use of redirected walking with distractors. The developed game has been used for two experiments which aim to provide some answers to the previously established research questions. In addition, a new redirection algorithm as well as a new forced reorientation technique has been developed for this thesis. Further information on these can be found in Chapter~\ref{chap:implementation}. 

\subsection{The Taxonomy of Distractors in Redirected Walking}
As part of the literature review in Chapter~\ref{chap:relatedWork}, a taxonomy detailing components and elements of distractors seen in the literature has been generated. The taxonomy is aimed towards providing classification, discussion points and analysis for research and development around distractors while allowing for extensions by future work. 

\section{Thesis Structure}
The thesis itself has a nested structure which corresponds to each research question. Each research question has an experiment dedicated to it and includes its own method, result and discussion components. The reasoning for this structure is that parts of the results for the experiment on noticeability are necessary for the second experiment which focuses on effectiveness. As such, it would be preferable to present the first experiment before moving on to the second. The overall structure of this thesis is as follows:

\begin{enumerate}
    \item Introduction
    \item Related Work and Background
    \item General Methods
    \item Implementation of Ensemble Retriever
    \item Experiment 1: Measuring the Noticeability of Distractors
    \begin{enumerate}
        \item Experiment Methods
        \item Experiment Results
        \item Experiment Discussion
    \end{enumerate}
    \item Experiment 2: Measuring the Effectiveness of Distractors
    \begin{enumerate}
        \item Experiment Methods
        \item Experiment Results
        \item Experiment Discussion
    \end{enumerate}
    \item Overarching Discussion
    \item Conclusion and Future Work
\end{enumerate}