\chapter{Introduction}
\label{chap:introduction}

\section{Topic Covered by The Thesis}
This thesis covers the topic of redirected walking and a particular subarea in this field known as virtual distractors (or simply distractors). A problem with current day room scale solutions in virtual reality is that users often do not have a large amount of space to move around in for these types of experiences. Redirected walking aims to mitigate this problem by unnoticeably redirecting the user while they walk around to create the illusion of a fully explorable virtual world~\cite{razzaque2001redirected}. By doing so, it is possible to make better use of the available physical space while still creating an immersive experience. Despite this optimisation, redirected walking by itself is not sufficient enough to properly redirect the user in smaller physical spaces~\cite{5072212,azmandian2015physical}. By engaging the user with distractors on the other hand, it is possible to increase the degree of unnoticeable redirection while still keeping a high subjective sense of presence~\cite{peck2009evaluation}. These distractors could be anything from activities in the virtual world to objects that can keep the user's attention.  


\section{Keywords}
Virtual Reality, Redirected Walking, Distractor, Distractors, Immersion, Subjective Sense of Presence, Computer Games, Games, Noticeability, Effectiveness, Detection, Detection Threshold, Detection Thresholds

\section{Problem Description}
Redirected walking by itself achieves full effectiveness in very large rooms which are unavailable to most users. As an example: it is necessary to have a room that can fit a circle with a radius of 22 meters to entirely redirect the user in an unnoticeable manner when using one type of redirection technique~\cite{5072212, azmandian2015physical}. It is not only unrealistic to assume that average end-users have access to such large rooms, but also challenging for many modern head-mounted displays to track areas of that size. In smaller physical spaces, the user is expected to be told by the software to reorient themselves a fair amount whenever they are close to the physical walls. These reorientation events can break the user's subjective sense of presence and does not necessarily contribute to an immersive virtual experience. 

\section{Justification, Motivation and Benefits}
The limitations of physical space mean that it is all the more important to make sure that reorientation events are as effective and unintrusive to the user as possible~\cite{azmandian2015physical}. The primary benefits of a good redirected walking solution lie with the end-user as it allows them to experience virtual reality in a more immersive manner as well as providing lower amounts of cybersickness compared to other forms of locomotion~\cite{razzaque2001redirected, peck2011evaluation}. It further allows the user to walk around in a virtual world that is larger than their available physical space, which could be seen as a crucial part of the experience. It also provides some benefits to the developers of virtual reality (VR) software as they do not have to rely on the limits of physical space to the same degree as they currently do. While there has been some research on the topic of distractors over the years, it is not a large field of research. As such, it would be beneficial to stimulate further research in this field as the problem of limited physical space is unlikely to disappear in the near future. The usage of distractors for the sake of reorientation is particularly useful as they focus on reorienting in a manner that aims to provide lower amounts of intrusion into the experience compared to other approaches.

\section{Research Questions}\label{research:questions}
While current research on distractors in redirected walking has focused on improving context sensitivity, there are some areas which are left unexplored. One such area is measuring how they affect the noticeability of redirection. Furthermore, if the highest unnoticeable redirection gains with distractors were estimated, it would also be interesting to see how effective the redirection could be in the context of a virtual environment like a game. The reason for using games as a context is that their interactive nature allows to broaden the design space of distractors in a manner that could be more engaging for the user. As such, the following research questions have been established:
\begin{description}
\item[$RQ_1$: ] How noticeable is redirected walking with distractors in a playful virtual environment?
\item[$RQ_2$: ] Given the highest unnoticeable gains, how effective is redirected walking with distractors in a playful virtual environment?
\end{description}

\section{Contributions}
With these research questions in mind, this master thesis has yielded a variety of contributions to the space of redirected walking. The following paragraphs provide a summary of these, categorised by contributions to research and secondary contributions to the redirected walking/VR communities. It should be noted that some background from Chapter~\ref{chap:relatedWork} may be necessary to understand the finer details of the contributions.

\subsection{Research Contributions}
For redirected walking research, this thesis has provided the following contributions: 

\subsubsection{The Taxonomy of Distractors in Redirected Walking}
As part of the literature review in Chapter~\ref{chap:relatedWork}, a taxonomy detailing components and elements of distractors seen in the literature has been generated. The taxonomy provides an empirically-supported classification, discussion points and analysis for the research and development around distractors while allowing for extensions by future work. 

\subsubsection{Providing New Insights With Exploratory Methods}
Given that the research field on distractors is still relatively small, this study has made use of rather exploratory approaches and methods. These methods have resulted in a large amount of potential future work for researchers as well as insights which might not have been possible to find with pre-established methods. Specifically for $RQ_1$, an incremental detection threshold estimation method has been developed with inspiration from Fuglestad's research~\cite{fuglestad2018redirected}. For $RQ_2$, a new effectiveness metric has been introduced, which consists of comparing the time taken for a distractor to align the user towards its goal relative to its active time. 

In general, the developed experiment environment can be considered as having a larger scope than existing work. This scope means that there are far more potential variables in play as the scenario is closer to the real world and as such, less controlled. As a result, it does create some challenges in terms of mitigating the influence of potentially unaccounted variables. By taking this approach though, it is possible to gain insights and find questions which might not have been possible otherwise. There has also been a significant focus on reproducibility to ensure that the exploratory methods that have been applied can be reproduced and reused by future research. 

\subsubsection{Experiment Results}
As part of providing answers to $RQ_1$ and $RQ_2$, two experiments have been conducted. The results from these experiments are considered as another contribution. The first of the experiments compared the differences in noticeability between two states: a walking state and a distractor battle state. No significant difference was found between the two for positive rotation gains. Despite this, it was significantly easier to notice negative rotation gains in the battle state. Furthermore, an adaptation effect towards positive rotation gains was observed.

The second experiment compared two effectiveness metrics between two separate conditions. The first of these consisted of using the ''Steer to Center'' (S2C) algorithm while walking and the developed ''Align Centre to Future'' (AC2F) algorithm during distractor battles. The second condition consisted of using S2C for both walking and battle states. No significant difference was found in terms of the mean number of resets or time taken to reach alignment towards physical room centre. Despite these results, it should be noted that the first condition experienced $15.8\%$ fewer failure cases in terms of completing alignment before the distractor was defeated. 

\subsection{Secondary Contributions}
Outside of the research contributions, there are a variety of secondary contributions which could be considered as beneficial to developers and researchers alike. These include:

\subsubsection{Providing an Example of State of the Art Distractor Usage}
The thesis has provided an example of how noticeable redirected walking is when using state of the art distractors. The reason for mentioning it has provided an example is that the design space for distractors is vast and as such, some implementations might be more effective than others. As more and more research into the field is generated, it is possible to improve the understanding of the effect that various variables might have on the results. 

\subsubsection{Contributing to a Young Field of Research}
Furthermore, the thesis has contributed to a field of research that currently is reasonably small and young. Due to the size and age of this field, it is beneficial to provide additional results that can point towards the effectiveness and noticeability of redirected walking with distractors. By doing so, it is possible to validate existing research as well as providing data that can be used for consideration by other researchers with interest in the topic. 

\subsubsection{An Openly Available VR Game: ''Ensemble Retriever''}
This thesis has also contributed an openly available VR game, titled ''Ensemble Retriever'' which makes use of redirected walking with distractors. The developed game has been used for two experiments to provide answers to the previously established research questions. The scope of the game can be considered as larger and more complex than existing work, making it a valuable example of redirected walking integration in larger projects. In addition, a new redirection algorithm, as well as a new resetting technique has been developed for this thesis. Further information on these can be found in Chapter~\ref{chap:implementation}. 

\section{Thesis Structure}
The thesis itself has a nested structure which corresponds to each research question. Each research question has an experiment dedicated to it and includes its own method, result and discussion components. The reasoning for this structure is that parts of the results for the experiment on noticeability are prerequisites for the second experiment, which focuses on effectiveness. The overall structure of this thesis is as follows:

\begin{enumerate}
    \item Introduction
    \begin{itemize}
        \item This chapter is an introduction to the topic, problem space, associated research questions and contributions.
    \end{itemize}
    \item Related Work and Background
    \begin{itemize}
        \item This chapter consists of a literature review which has been written with the established research questions in mind. It also provides the necessary background for understanding redirected walking.
    \end{itemize}
    \item General Methods
    \begin{itemize}
        \item This chapter details general methods which apply to the thesis as a whole.
    \end{itemize}
    \item Implementation of Ensemble Retriever
    \begin{itemize}
        \item This chapter provides abstracted implementation details for the developed Ensemble Retriever game, and its corresponding redirected walking functionality. The overarching game design of the game and its usage of distractors is also detailed.
    \end{itemize}
    \item Experiment 1: Measuring the Noticeability of Distractors
    \begin{enumerate}
        \item This chapter is focused on an experiment which tests for differences in noticeability between a walking state and a distractor battle state when playing Ensemble Retriever. 
    \end{enumerate}
    \item Experiment 2: Measuring the Effectiveness of Distractors
    \begin{enumerate}
        \item This chapter is focused on an experiment that compares effectiveness metrics in terms of the mean number of resets and time taken for alignment towards physical room centre. This comparison is made between a control condition and an experimental one that employs a newly developed redirection algorithm. 
    \end{enumerate}
    \item Overarching Discussion
    \begin{itemize}
        \item This chapter takes a general look at the thesis as a whole and discusses various limitations, future improvements and challenges which have been observed throughout this research.
    \end{itemize}
    \item Conclusion and Future Work
    \begin{itemize}
        \item The thesis then concludes on the research that has been conducted and ends with a large amount of future work which would benefit from further exploration.
    \end{itemize}
\end{enumerate}