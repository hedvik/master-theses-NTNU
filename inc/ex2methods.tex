\chapter{Experiment 2: Effectiveness of Distractors}

\section{Method}
\todo{Important thing to consider: How to structure the control condition. Does it consist of 0 redirection, the nondistractor thresholds that were calculated in experiment 1 or S2C only instead of S2C+AC2F(most likely)?}

\subsection{Data Collection}
   * For each distractor:
      * distractor++
      * distractorType
      * Time taken to reach alignment
      * Time taken until distractor was defeated
      * (If the gains are high enough, they might be able to align very early)
         * As such you can either: 
            * Make use of lower gains to improve potential comfort and decrease noticeability for those who have an easier time noticing gains
            * Decrease the time needed to finish a distractor
   * For each reset
      * reset++
      * was an distractor active?
         * What distractor was it?
         * time since distractor activation
      * time since last reset
   * Experiment time between tutorial finish and starting the boss fight with the mountain king
      * (To be able and normalise number of resets relative to time spent walking around in the open world)
      * Should probably not count time playing with distractors as they might choose shield upgrade first. Recording time spent walking is more reasonable.
   * Virtual + Real Path
   * Chosen Player Upgrades and their levels
      * Higher damage upgrades means that less time is spent defeating enemy distractors. It could correlate with higher reset counts if the redirection is insufficient as a result of the distractor finishing earlier.
   
\subsection{Sample}