\chapter{Conclusion}\label{chap:conclusion}
* Summary
* This was a pretty exploratory study into relatively unexplored waters
   * Resulted in a lot of future work!

What were the findings on different levels?
   * What has the research provided insights on in
      * The specific distractor/redirection level
      * Redirected walking as a whole
         * A new resetter
         * Potential optimisation of heuristic redirection algorithms where the user is standing still
         * Open Source Project using RDW
      * General impact in VR
      * ETC


\section{Future Work}\label{sec:futurework}
Where would the project go from here. While this section focuses on areas of future research that were detected through discussion and results, it may not guarantee that there are not already existing papers or research already looking into this. The literature sample for this thesis is not comprehensive enough to cover the entirety of the redirected walking space, but serves as an approximation in relation to the main topic of distractors within redirected walking. 

* What can be improved
* AC2F
   * Could become more generic by adding support for walking around
   * Smoothing can be improved
   * Potential studies into the effect of smoothing solutions on detection thresholds

* Looking deeper into the asymmetry between positive and negative rotation gains. 

* Identifying the variables that affect why there is a significant difference that widens the detection thresholds when walking. (The data variability most likely has a pretty strong effect on why it is like it is for negative rot gains)
   * Is expectation mismatch relevant when fighting as you do not have quite a set target to look at when walking?
      * Goal directed looking
   * Frame of reference
      * Can the moving distractor be considered a frame of reference for redirection?
      * This is of course anecdotal, but some participants mentioned using the glowing mushrooms as a frame of reference, but S2C still goes lower than AC2F on average on negative gains
      * Optical flow/visual density generated by distractors
         * Since some participants mentioned they used the glowing mushroom as a reference point to notice redirection, would a distractor work as one of these reference points?
   * Differences in algorithms
      * Smoothing solution for AC2F may be an effect
      * If alignment as a concept is disabled from AC2F you will be able to apply more rotation gains to the participants during battles
         * This would help with the asymmetry in number of detections between algorithms, more time would likely be spent with AC2F than S2C in terms of potential space to detect gains
         * Might also result in stronger adaptation effects
            * We would need further studies into how temporarily disabling gains affects adaptation effects

* Improving data recording
   * improving accuracy of variables, automating data post processing by handling this on the software side. 

* Other interesting pathways
   * Having people's first experience with VR be with redirected walking
      * They might normalise the redirected walking elements
         * \todo{Will have to look deeper into the data on people with no experience and how many detections they did}
   * Effectiveness of disabling redirected walking towards the end of an experience to help the user get accustomed to normal head rotation before taking off the HMD
   
   * Looking deeper into why reorientations using S2C may be better off doing a reorientation that is parallel to walls at times rather than towards the room centre
      * Could be simulated with the toolkit for example to see whether relative effectiveness increases compared to 
      
   * What is optimal in terms of cybersickness when using distractors? Stronger gains for short periods of time with no redirection breaks when aligned or lower gains throughout the entire distractor usage
   
   * How does temporarily disabling gains affect adaptation effects?
   
   * Looking deeper into how prior VR experience affects noticability. 
   
   * effective and short Individual DT calibration methods